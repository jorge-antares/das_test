\documentclass[aspectratio=169, 10pt]{beamer}

\usetheme{CambridgeUS}
\usecolortheme{dolphin}
\setbeamertemplate{navigation symbols}{}
\setbeamertemplate{footline}[frame number]

\usepackage{booktabs}
\usepackage{listings}
\usepackage{xcolor}
\usepackage{tikz}

\lstset{
    basicstyle=\ttfamily\small,
    keywordstyle=\color{blue},
    commentstyle=\color{gray},
    backgroundcolor=\color{gray!10},
    frame=single,
    breaklines=true,
    language=Python
}

\title{Data Analytics Specialist Test}
\subtitle{Cleaning, Validation, and Analysis of a Historical Aviation Crashes}
\author{Jorge A. Garcia}
\date{February 23, 2026}


\begin{document}

% ── Slide 1: Title ────────────────────────────────────────────────────────────
\begin{frame}
    \titlepage
\end{frame}

% ── Slide 2: Outline ────────────────────────────────────
\begin{frame}{Outline}
    \tableofcontents
\end{frame}

% ── Slide 3: About me ────────────────────────────────────
\section{About me}
\begin{frame}{About me}
    % List key points about your background, motivation, and approach to data analytics.
    \begin{itemize}
        \item Quantitative modeler with 10+ years of experience:
        % Sub list
        \begin{itemize}
            \item Optimization (e.g., LP, ML)
            \item Agent-based simulation
            \item Physics-based modeling
            \item Macroeconomic models
        \end{itemize}
        \vspace{0.3cm}
        \item Currently, developer of web-apps for science and decision-making (https://modelbridgelabs.com/).
        \vspace{0.3cm}
        \item Goals:
            \begin{itemize}
                \item Apply data analytics and modeling to inform decision-making and uncover actionable insights.
                \item Learn and grow as a professional by taking on new challenges and responsibilities.
            \end{itemize}
    \end{itemize}

\end{frame}


% ── Slide 4: Dataset and Pipeline Overview ────────────────────────────────────
\section{General approach}
\begin{frame}{General approach}
    \begin{columns}[T]
        \column{0.5\textwidth}
            \begin{enumerate}
                \item Exploratory analysis
                \item Structure design
                \item Data cleaning
                \item Automated validation
                \item Data profiling
                \item Version control, CI, testing
                \item Iterative refinement
            \end{enumerate}
        \column{0.5\textwidth}
        % Figure
        \begin{figure}
            \centering
            \includegraphics[width=0.8\textwidth]{img/structure.png}
            \caption{Project structure.}
        \end{figure}
    \end{columns}
\end{frame}

% ── Slide 5: File description ────────────────────────────────────
\section{Project structure}
\begin{frame}{Project structure}
    \begin{itemize}
        \item \texttt{utilspc/}
        \begin{itemize}
            \item \texttt{cleanfun.py} --- Parsing functions for cleaning specific fields.
            \item \texttt{cleanerclass.py} --- Python classes for SQLite connection and cleaning.
        \end{itemize}
        \item \texttt{analytics/}
        \begin{itemize}
            \item \texttt{validation.py} --- validation checks and profiling report generation.
            \item \texttt{analysis.py} --- analysis functions for generating reports and insights.
        \end{itemize}
        \item \texttt{main.py} --- main script to run the cleaning, validation, and analysis pipeline.
        \item \texttt{test.py} --- test for type checking, cleaning functions, and validation checks.
        \item \texttt{output}
        \begin{itemize}
            \item \texttt{metadata.csv} --- column-level profile of the cleaned dataset.
            \item \texttt{data\_profile.txt} --- full analysis report of the cleaned dataset.
            \item \texttt{cleaned\_plane\_crashes.db} --- cleaned SQLite database.
            \item \texttt{validation\_report.txt} --- detailed validation report with check results and failure details.
        \end{itemize}
    \end{itemize}
\end{frame}


% ── Slide 6: Data Cleaning ────────────────────────────────────────────────────
\section{Data Cleaning}
\begin{frame}{Date, Time and Location}
    \begin{itemize}
        \item \textcolor{blue}{\texttt{date}}: Parse `DD-Mon-YY' to `YYYY-MM-DD', or None/Null on failure. Year values < 26 are treated as 2000s, otherwise 1900s.
        \vspace{0.3cm}
        \item \textcolor{blue}{\texttt{time}}: Normalise to `HH:MM' (24-hour). Strips prefix `c' with optional space.\\
            Handles 4-digit integers (e.g. `1730' to `17:30'). Returns None for unparseable values.
        \vspace{0.3cm}
        \item \textcolor{blue}{\texttt{location}} Returns a tuple (`first', `last') where `first' is the location excluding the country (or None) and `last' is the country name (or None). Creates field \textcolor{blue}{\texttt{country}}.
    \end{itemize}

\end{frame}

\begin{frame}{Operator, AC type, aboard and fatalities}
    \begin{itemize}
        \item \textcolor{blue}{\texttt{operator}}: Strip whitespace, map `?' to NULL and nullify values that are clearly not airline operators:\\
            \begin{itemize}
                \item Pure serial / registration codes, e.g. ``'46826/109'``.
                \item Aircraft manufacturer + model designator entries that were incorrectly placed in the operator column, e.g. `Boeing KC-135E', `Lockheed AC-130H Hercules'.
            \end{itemize}
        \vspace{0.3cm}
        \item \textcolor{blue}{\texttt{ac\_type}}:
            \begin{itemize}
                \item Strip whitespace and map `?' to NULL.
                \item Remove vehicle categories, e.g. `(flying boat)', `(airship)', `(amphibian)'.
                \item Remove extra spaces.
                \item Strip any residual leading `/' trailing whitespace.
                \item Return None is empty
            \end{itemize}

        \item \textcolor{blue}{\texttt{aboard, fatalities}}: Returns (total, passengers, crew) as int or None.
    \end{itemize}

\end{frame}


% ── Slide 7: Data Validation ─────────────────────────────────────────────────
\section{Data Validation}
\begin{frame}{Data Validation}

    \textbf{Automated checks run after every cleaning pass:}

    \bigskip

    \begin{columns}[T]
        \column{0.5\textwidth}
        \textbf{Structure}
        \begin{itemize}
            \item Schema: all 18 expected columns with correct declared types
            \item Python-level type consistency per column
        \end{itemize}

        \bigskip

        \textbf{Format}
        \begin{itemize}
            \item Dates match \texttt{YYYY-MM-DD}; range 1908--2018
            \item Times match \texttt{HH:MM} (24-hour)
            \item All numeric columns $\geq 0$
        \end{itemize}

        \column{0.5\textwidth}
        \textbf{Cross-column consistency}
        \begin{itemize}
            \item \texttt{aboard\_pax + aboard\_crew = aboard\_total}
            \item \texttt{fat\_pax + fat\_crew = fatalities\_aboard}
            \item \texttt{fatalities\_aboard} $\leq$ \texttt{aboard\_total}
            \item \texttt{fatalities\_total = fatalities\_aboard + ground}
        \end{itemize}

        \bigskip

        \textbf{Duplicates}
        \begin{itemize}
            \item Rows sharing \texttt{(date, operator, route)} flagged as potential duplicates
        \end{itemize}
    \end{columns}

    \bigskip
    \small
    Results reported as \textbf{PASS / WARN / FAIL} per check.

\end{frame}


% ── Slide 8: Analysis Sections ────────────────────────────────────────────────
\section{Analysis}
\begin{frame}{Analysis}

    \textbf{Five sections produced in the profiling report:}

    \bigskip

    \begin{enumerate}
        \setlength\itemsep{6pt}
        \item \textbf{Data Profiling} --- NULL rate, unique value count, and type per column.

        \item \textbf{Descriptive Statistics} --- Aggregate fatality rate (fatalities\,/\,aboard), survival rate, ground casualties, and crew vs.\ passenger fatality split.

        \item \textbf{Trend Analysis} --- Crashes and fatalities per decade and per year (ASCII bar chart); top 15 operators by crash count; most dangerous aircraft types by fatality rate (min.\ 10 incidents).

        \item \textbf{Geographic Analysis} --- Top 20 countries/regions and specific crash sites by incident count.

        \item \textbf{Data Quality} --- Mismatched totals, rows where fatalities exceed aboard count, duplicate \texttt{(date, operator, route)} groups, and registration reuse across aircraft types.
    \end{enumerate}

\end{frame}

% ── Slide 9: Conclusion and Future Work ─────────────────────────────────────
\section{Summary and Improvements}
\begin{frame}{Summary and Future Work}
    \textbf{My approach:}
    \begin{itemize}
        \item Develop a modular and reusable structure.
        \item Implement programming best practices: version control, testing, CI/CD, and documentation.
        \item Explore ways to integrate data products with decision-making processes and end-user applications.
    \end{itemize}

    \bigskip

    \textbf{Improvements:}
    \begin{itemize}
        \item Enhance location parsing to better handle edge cases and extract more granular geographic information.
        \item Implement additional validation checks for outliers and temporal inconsistencies (e.g., crashes dated before the Wright brothers' first flight).
        \item Standardize operator names, AC types, and geographic scales.
        \item Complement data with external reference datasets (e.g., GIS, weather, economic indicators).
    \end{itemize}

\end{frame}

\end{document}
